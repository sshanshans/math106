\documentclass[12pt,twoside]{article}
\usepackage{amssymb, amsmath, mathrsfs, epsfig}
\usepackage{fancyhdr}
\usepackage{tikz}
\setlength{\voffset}{-1in}
\setlength{\topmargin}{0in}
\setlength{\textheight}{9.5in}
\setlength{\textwidth}{6.5in}
\setlength{\hoffset}{0in}
\setlength{\oddsidemargin}{0in}
\setlength{\evensidemargin}{0in}
\setlength{\marginparsep}{0in}
\setlength{\marginparwidth}{0in}
\setlength{\headsep}{0.25in}
\setlength{\headheight}{0.5in}
\pagestyle{fancy}

\fancyhead[LO,LE]{Math 106L}
\fancyhead[RO,RE]{Problem set 1}
\chead{\textbf{Trig Identities and Limits}}
\cfoot{}
\fancyfoot[LO,LE]{}
\fancyfoot[RO,RE]{Page \thepage\ of \pageref{LastPage}}
\renewcommand{\footrulewidth}{0.5pt}
\parindent 0in

\newcommand{\ben}{\begin{enumerate}}
\newcommand{\een}{\end{enumerate}}
\newcommand{\R}{\mathbb{R}}
\newcommand{\Sp}{\vspace{.5in}}
\newcommand{\defn}{\paragraph*{Definition}}
\newcommand{\example}{\paragraph*{Example}}
\newcommand{\examples}{\paragraph*{Examples}}
\newcommand{\qns}{\paragraph*{Questions}}
\newcommand{\qn}{\paragraph*{Question}}
\newcommand{\blank}[1]{\underline{\hspace{#1}}}
\newcommand{\dsst}{\displaystyle}

\begin{document}

We defined $\tan \theta=\frac{\sin\theta}{\cos\theta}$. The following questions help you think a little more about $\tan$. Yes, summer is not too far away. :) 

\begin{enumerate}
      \item What is the period of $\tan x$
      
      The period of $\tan x$ is $\pi$. (1pt)
      \item For what values of $x$ is $\tan x$ positive?  Negative?  Zero? (1pt)
     
      $\tan x$ is 0 when $x = k\pi$ for any integer $k$; $\tan x$ is positive at $(0 + k\pi, \frac{\pi}{2} + k\pi)$ for any integer $k$; $\tan x$ is negative at $(-\frac{\pi}{2} + k \pi, 0 + k \pi)$ for any integer $k$. 
      
      [Suggested by Janelle, Chris Long] Values of $x$ in Quadrants I and III yield a positive $\tan x$, values of $x$ in Quadrants II and IV yield negative $tanx$, and the $x$ values of -1 and 1 yield 0 for $\tan x$. 
      
       \item Where does $\tan x$ have vertical asymptotes? (1pt)
       
       $\tan x$ has vertical asymptotes at $\frac{\pi}{2} + k \pi$ for any integer $k$. 
       
      \item Does $\dsst\lim_{x\to\frac{\pi}{2}} \tan x$ exist?  Why or why not? (1pt)
      
     No it doesn't. Because $\tan x$ goes to positive infinity as $x$ goes to $\frac{\pi}{2}$. 
      \item Does $\tan (|x|) = | \tan(x) |$? Why or why not? (1pt)
      
      No. $\tan (|\frac{3\pi}{4}|)$ is negative, but $|\tan (\frac{3\pi}{4})|$ is positive. 
      \item Show $\tan^2 x + 1 = \sec^2 x$. (3pt)
      
      This is equivalent to show $\frac{\sin^2(x)}{\cos^2(x)} + 1 = \frac{1}{\cos^2(x)}$ (1pt). Rewrite the left hand side, we have $\frac{\sin^2(x)+\cos^2(x)}{\cos^2(x)} $ (1pt) . Since $\sin^2(x)+\cos^2(x) = 1$, we obtained the right hand side which is $\frac{1}{\cos^2(x)}$ (1pt). 
      
      \item Show $\cos x \tan x \csc x = 1$.  (2pt)
      
      Note that the left hand side can be written as $\cos x \frac{\sin x}{\cos x} \frac{1}{\sin(x)}$ (1pt). Simplify the fraction we get 1 (1pt). 
\end{enumerate}


\end{document}
