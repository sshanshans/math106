\documentclass[12pt,twoside]{article}
\usepackage{amssymb, amsmath, mathrsfs, epsfig}
\usepackage{fancyhdr}
\usepackage{tikz}
\setlength{\voffset}{-1in}
\setlength{\topmargin}{0in}
\setlength{\textheight}{9.5in}
\setlength{\textwidth}{6.5in}
\setlength{\hoffset}{0in}
\setlength{\oddsidemargin}{0in}
\setlength{\evensidemargin}{0in}
\setlength{\marginparsep}{0in}
\setlength{\marginparwidth}{0in}
\setlength{\headsep}{0.25in}
\setlength{\headheight}{0.5in}
\pagestyle{fancy}

\fancyhead[LO,LE]{Math 106L}
\fancyhead[RO,RE]{Problem set 4}
\chead{\textbf{DE and Riemann Sums}}
\cfoot{}
\fancyfoot[LO,LE]{}
\fancyfoot[RO,RE]{Page \thepage\ of \pageref{LastPage}}
\renewcommand{\footrulewidth}{0.5pt}
\parindent 0in

\newcommand{\ben}{\begin{enumerate}}
\newcommand{\een}{\end{enumerate}}
\newcommand{\R}{\mathbb{R}}
\newcommand{\Sp}{\vspace{.5in}}
\newcommand{\defn}{\paragraph*{Definition}}
\newcommand{\example}{\paragraph*{Example}}
\newcommand{\examples}{\paragraph*{Examples}}
\newcommand{\qns}{\paragraph*{Questions}}
\newcommand{\qn}{\paragraph*{Question}}
\newcommand{\blank}[1]{\underline{\hspace{#1}}}
\newcommand{\dsst}{\displaystyle}

\begin{document}
Please turn in the following problems. 

5.1: 26 \\

5.2: 11, 14, 31, 35\\

5.3: 2, 6, 14, 17(a), 17(b)(ii), 24 \\

Chapter 5 Review: 40 \\

Past exam questions: 

I. Consider the graph of the function $g(x)=\sin x$ between $x=0$ and $x=\pi$ radians.

\begin{enumerate}
 \item Calculate the left-hand sum with $n=3$ rectangles (a.k.a. $LHS_3$) for the area under the graph of $f(x)$ over this domain.  You must write out your calculation in full.  No calculator.
 \item Draw one or more pictures to explain why the quantity $$\frac{LHS_3 + RHS_3}{2}$$ must equal the number you calculated in part (a).  Only pictures and words (and/or symbols) are acceptable.  No further calculations.
 \item Without doing any further calculations, fill in the blank with one of the symbols $<$, $>$, or $=$: $$MPS_3\mbox{ \blank{0.25in} }RHS_3.$$
 \item \emph{Briefly} explain your answer to part (c).  (Hint: You may find it helpful to refer to your answer to part (b) of this question, as well as to the concavity of $g(x)$.)
\end{enumerate}


II.  Consider the Riemann sum $$\sum_{k=0}^{9999} \left(e^{-(9+0.002k)^2} - 6(9+0.002k)^2\right)\times 0.002$$
\begin{enumerate}
 \item Is this a left hand sum, right hand sum, midpoint sum, or something else?
 \item How many terms are in the sum?  (i.e. what is $n$?
 \item By first finding $\Delta x$, $a$, and $b$, write down the definite integral this sum approximates.
 \item Is the following sum greater than, less than, or equal to the one above?  Justify your answer.
$$\sum_{k=0}^{99999} \left(e^{-(9+0.0002k)^2} - 6(9+0.0002k)^2\right)\times 0.0002$$\vfill
\end{enumerate}



\end{document}
