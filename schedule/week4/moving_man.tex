\documentclass[12pt,twoside]{article}
\usepackage{amssymb, amsmath, mathrsfs, epsfig}
\usepackage{fancyhdr}
\usepackage{tikz}
\setlength{\voffset}{-1in}
\setlength{\topmargin}{0in}
\setlength{\textheight}{9.5in}
\setlength{\textwidth}{6.5in}
\setlength{\hoffset}{0in}
\setlength{\oddsidemargin}{0in}
\setlength{\evensidemargin}{0in}
\setlength{\marginparsep}{0in}
\setlength{\marginparwidth}{0in}
\setlength{\headsep}{0.25in}
\setlength{\headheight}{0.5in}
\pagestyle{fancy}

\fancyhead[LO,LE]{Math 106L}
\fancyhead[RO,RE]{Example}
\chead{\textbf{Moving man}}
\cfoot{}
\fancyfoot[LO,LE]{}
\fancyfoot[RO,RE]{Page \thepage\ of \pageref{LastPage}}
\renewcommand{\footrulewidth}{0.5pt}
\parindent 0in

\newcommand{\ben}{\begin{enumerate}}
\newcommand{\een}{\end{enumerate}}
\newcommand{\R}{\mathbb{R}}
\newcommand{\Sp}{\vspace{.5in}}
\newcommand{\defn}{\paragraph*{Definition}}
\newcommand{\example}{\paragraph*{Example}}
\newcommand{\examples}{\paragraph*{Examples}}
\newcommand{\qns}{\paragraph*{Questions}}
\newcommand{\qn}{\paragraph*{Question}}
\newcommand{\blank}[1]{\underline{\hspace{#1}}}
\newcommand{\dsst}{\displaystyle}

\begin{document}
Suppose a man moves along a line with acceleration $\frac{F}{m} = 1 m/s^2$. Suppose he is standing 2 meters to the right of a reference point at time 0 seconds, moving to the left at 1m/s. 

\begin{enumerate}
\item Let's set up an initial value problem to find expressions for his position and velocity. 

Let $s(t)$ denote the position of the man at time $t$, where $t$ is measured in seconds and $s$ is measured in meters. Let $s'(t)$ denote the velocity of the man at time $t$. Assume right is positive. 
\begin{align*}
s''(t) = 1 \\
s(0) = 2 \\
s'(0) = -2
\end{align*}

To find the solution for this initial value problem, we first antidifferentiate $s''(t)$. We have $s'(t) = t + b$. Plugging in $s'(0) = -2$, we have $b = -2$ and thus $s'(t) = t-2$. 

Next antidifferentiate again for $s'(t)$. We then have $s'(t) = \frac{t^2}{2} -2t +c$. Plugging in $s(0) = 2$, we have $c = 2$. Therefore, $s(t) =  \frac{t^2}{2} -2t +2$. 

\item Describe the motion of this man. 
\begin{itemize}
\item At time 0 seconds, the man is 2 meters to the right of the reference point. It is moving to the left at a speed of 2 m/s. 
\item The velocity increases throughout the entire motion, since the acceleration is a positive constant. 
\item The man continues moves to the left with decreasing speed for 2 seconds. 
\item At time 2 seconds, the man comes to an instantaneous stop right at the reference point. 
\item Following that, the man moves to the right with increasing speed. 
\item The object passes the reference point at time 2 seconds. 
\end{itemize}
\end{enumerate}

In general, when you are asked to describe the motion on an object, follow the template below. 
\begin{itemize}
 \item Note the object's initial position, speed, and direction of motion.
 \item Next, note how long it moves in the initial direction, and whether it is slowing down or speeding up.
 \item Then write down every point at which it stops instantaneously, the subsequent direction of motion, and whether it is speeding up or slowing down following the instantaneous halt.
 \item Lastly, note all the times it passes the reference point (position 0), if at all.
\end{itemize}

\end{document}