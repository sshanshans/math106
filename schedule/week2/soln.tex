\documentclass[12pt,twoside]{article}
\usepackage{amssymb, amsmath, mathrsfs, epsfig}
\usepackage{fancyhdr}
\usepackage{tikz}
\setlength{\voffset}{-1in}
\setlength{\topmargin}{0in}
\setlength{\textheight}{9.5in}
\setlength{\textwidth}{6.5in}
\setlength{\hoffset}{0in}
\setlength{\oddsidemargin}{0in}
\setlength{\evensidemargin}{0in}
\setlength{\marginparsep}{0in}
\setlength{\marginparwidth}{0in}
\setlength{\headsep}{0.25in}
\setlength{\headheight}{0.5in}
\pagestyle{fancy}

\fancyhead[LO,LE]{Math 106L}
\fancyhead[RO,RE]{Problem set 2}
\chead{\textbf{Trig Derivatives and Inverse Trigs}}
\cfoot{}
\fancyfoot[LO,LE]{}
\fancyfoot[RO,RE]{Page \thepage\ of \pageref{LastPage}}
\renewcommand{\footrulewidth}{0.5pt}
\parindent 0in

\newcommand{\ben}{\begin{enumerate}}
\newcommand{\een}{\end{enumerate}}
\newcommand{\R}{\mathbb{R}}
\newcommand{\Sp}{\vspace{.5in}}
\newcommand{\defn}{\paragraph*{Definition}}
\newcommand{\example}{\paragraph*{Example}}
\newcommand{\examples}{\paragraph*{Examples}}
\newcommand{\qns}{\paragraph*{Questions}}
\newcommand{\qn}{\paragraph*{Question}}
\newcommand{\blank}[1]{\underline{\hspace{#1}}}
\newcommand{\dsst}{\displaystyle}

\begin{document}

The following questions help you think a little more about $\cos(x)$. 

If $g(x) = \cos(x)$, 
\begin{enumerate}
      \item Use the definition of the derivative to show that $g'(x) = -\sin(x)$. [4pt]
      
      From the definition of derivatives, we have 
      \begin{align*}
      g'(x) &= \lim_{h \rightarrow 0} \frac{\cos(x+h) - \cos(x)}{h}  ~~~(1) \\
      & = \lim_{h \rightarrow 0} \frac{ \cos(x)\cos(h) - \sin(x) \sin(h) - \cos(x)}{h} ~~~(1)  \\
      & = \cos(x)  \lim_{h \rightarrow 0} \frac{\cos(h) -1}{h} - \sin(x)  \lim_{h \rightarrow 0} \frac{\sin(h)}{h} ~~~(1) \\
      & = -\sin(x)
      \end{align*}
      since $\lim_{h \rightarrow 0} \frac{\cos(h) -1}{h} = 0 $ and $\lim_{h \rightarrow 0} \frac{\sin(h)}{h}$ is 1. (1)
      
      \item True of False: $g''(x) = -g(x)$. Explain your answer. [1pt]
     
      True, since $g''(x) = -\cos(x) = -g(x)$. 
      
      \item Graph $y = \cos(x)$ on $[-2\pi, 2\pi]$. 
      
      \item Why is $\cos(x)$ not invertible on $[-2\pi, 2\pi]$.? [1pt]
     
      $\cos(x)$ is not one-to-one on $[-2\pi, 2\pi]$. 
      
      \item What is the simplest domain on which $\cos(x)$ is invertible? [1pt]
      
      $\cos(x)$ is invertible on $[0, \pi]$. 
      
      \item Let's call the inverse of $\cos(x)$ on that domain $\cos^{-1}(x)$, or $\arccos x$. What is the domain of $\cos^{-1}(x)$? The range? [1pt]
      
      The domain of $\cos^{-1}(x)$ is $[-1, 1]$, and the range is $[0, \pi]$.
    
      \item What is $\cos(\cos^{-1} x)$? For which values of $x$ is that true? [1pt]
      
      $\cos(\cos^{-1} x)$ is $x$ for $x$ in $[-1, 1]$.
      
      \item What is $\cos^{-1} (\cos x)$? For which values of $x$ is that true? [1pt]
      
      $\cos^{-1} (\cos x)$ is $x$ for all $x$.
      
      \item Sketch a graph of $y = \cos^{-1} x$. 
\end{enumerate}


\end{document}
